\documentclass{zjureport}
\usepackage{hyperref}
% =============================================
% Part 0 Edit the info
% =============================================

\major{计算机体系结构}
\name{卞留念}
\title{实验报告}
\stuida{201828013229131}
\stuidb{2018E8013261055}
\college{计算机学院}
\date{\zhtoday}
\lab{寝室}
\course{计算机网络}
\instructor{谢高岗}
\expname{静态路由转发实验}
\exptype{设计实验}
\partner{卫一宁}

\begin{document}
% =============================================
% Part 1 Header
% =============================================
\makecover
\makeheader

% =============================================
% Part 2 Main document
% =============================================

\section{实验要求}
  本实验要求学生在已有代码基础上,完善其中的TODO部分,实现路由器的IP查找转发、ARP请求和应答、ARP缓存管理、发送ICMP消息等功能。
\section{实验内容和步骤}

  \subsection{实验内容一}

      \begin{enumerate}
          \item 运行给定网络拓扑(router\_topo.py)
          \item Ping 10.0.1.1 (r1),能够ping通
          \item Ping 10.0.2.22 (h2),能够ping通
          \item Ping 10.0.3.33 (h3),能够ping通
          \item Ping 10.0.3.11,返回ICMP Destination Host Unreachable
          \item Ping 10.0.4.1,返回ICMP Destination Net Unreachable
      \end{enumerate}

  \subsection{实验内容二}


      \begin{enumerate}
          \item 构造一个包含多个路由器节点组成的网络,手动配置每个路由器节点的路由表,有两个终端节点,通过路由器节点相连,两节点之间的跳数不少于3跳,手动配置其默认路由表
          \item 终端节点ping每个路由器节点的入端口IP地址,能够ping通
          \item 在一个终端节点上traceroute另一节点,能够正确输出路径上每个节点(入端口)的IP信息
      \end{enumerate}

\section{主要仪器设备}
  计算机,Mininet 软件,Wireshark 软件

\section{实验步骤}
  \subsection{安装mininet与wireshark软件}
      下载并安装mininet与wireshark等软件。
  \subsection{编写代码TODO部分}
      根据所学知识完成代码的TODO部分。
  \subsection{完成实验内容}
      编译代码生成程序,在路由器节点上运行此程序,依次完成实验内容。

\section{实验过程}
  \subsection{安装mininet与wireshark软件}
     在Ubuntu下输入sudo apt install mininet 与 sudo apt install build-essential xterm wireshark ethtool iperf traceroute iptables arptables命令进行软件安装,安装完成后,运行sudo mn验证mininet是否正确安装,验证结果如图~\ref{fig:install} 所示。
           \begin{figure}[!htbp]
               \centering
               \includegraphics[width=0.7\linewidth]{figures/02.jpg}
               \caption{mininet安装成功}
               \label{fig:install}
           \end{figure}

  \subsection{编写arpcache\_lookup函数}
      \lstinputlisting[language=C]{code/arpcache_lookup.c}

  \newpage
  \subsection{编写arpcache\_append\_packet函数}
      \lstinputlisting[language=C]{code/arpcache_append_packet.c}

  \newpage
  \subsection{编写arpcache\_insert函数}
    \lstinputlisting[language=C]{code/arpcache_insert.c}

  \newpage
  \subsection{编写arpcache\_sweep函数}
    \lstinputlisting[language=C]{code/arpcache_sweep.c}

  \newpage
  \subsection{编写arp\_send\_request函数}
    \lstinputlisting[language=C]{code/arp_send_request.c}

  \newpage
  \subsection{编写arp\_send\_reply函数}
    \lstinputlisting[language=C]{code/arp_send_reply.c}

  \newpage
  \subsection{编写handle\_arp\_packet函数}
    \lstinputlisting[language=C]{code/handle_arp_packet.c}

  \subsection{编写iface\_send\_packet\_by\_arp函数}
    \lstinputlisting[language=C]{code/iface_send_packet_by_arp.c}

  \newpage
  \subsection{编写longest\_prefix\_match函数}
    \lstinputlisting[language=C]{code/longest_prefix_match.c}

  \subsection{编写handle\_ip\_packet函数}
    \lstinputlisting[language=C]{code/handle_ip_packet.c}

  \newpage
  \subsection{编写icmp\_send\_packet函数}
    \lstinputlisting[language=C]{code/icmp_send_packet.c}

  \subsection{编写ip\_forward\_packet函数}
    \lstinputlisting[language=C]{code/ip_forward_packet.c}



  \newpage
  \section{实验结果与分析}
  \subsection{实验一}
      \begin{enumerate}
          \item 运行router\_topo.py生成的网络如图~\ref{fig:rtpy} 所示。
                \begin{figure}[!htbp]
                    \centering
                    \includegraphics[width=0.7\linewidth]{figures/rtpy.png}
                    \caption{router\_topo.py生成的网络}
                    \label{fig:rtpy}
                \end{figure}

          \item 实验一(2)-(6)ping结果如图~\ref{fig:e1result} 所示。
                \begin{figure}[!htbp]
                    \centering
                    \includegraphics[width=0.7\linewidth]{e1result.png}
                    \caption{实验一(2)-(6)ping程序结果}
                    \label{fig:e1result}
                \end{figure}

         \item 实验一(2)-(6)路由程序运行结果如图~\ref{fig:e1program}所示。
                 \begin{figure}[!htbp]
                     \centering
                     \includegraphics[width=0.7\linewidth]{e1program.png}
                     \caption{实验一(2)-(6)路由程序运行结果}
                     \label{fig:e1program}
                 \end{figure}

      \end{enumerate}

  \newpage
  \subsection{实验二}
      \begin{enumerate}
          \item 根据要求编写router\_topo2.py, 代码如下。
                \lstinputlisting[language=Python]{code/router_topo2.py}

          \item router\_topo2.py生成的网络如图~\ref{fig:rtpy2} 所示。
                \begin{figure}[!htbp]
                    \centering
                    \includegraphics[width=0.9\linewidth]{figures/rtpy2.png}
                    \caption{router\_topo2.py生成的网络}
                    \label{fig:rtpy2}
                \end{figure}

          \item 实验二(2)ping结果如图~\ref{fig:e2pingresult} 所示。
                \begin{figure}[!htbp]
                    \centering
                    \includegraphics[width=0.7\linewidth]{figures/e2result1.png}
                    \caption{实验二(2)ping程序结果}
                    \label{fig:e2pingresult}
                \end{figure}

         \item 实验二(2)traceroute结果如图~\ref{fig:e2tracerouteresult}所示。
               \begin{figure}[!htbp]
                   \centering
                   \includegraphics[width=0.7\linewidth]{figures/e2result2.png}
                   \caption{实验二(3)traceroute程序结果}
                   \label{fig:e2tracerouteresult}
               \end{figure}

         \item 实验二(2)-(3)路由1程序运行结果如图~\ref{fig:e2program1}所示。
                 \begin{figure}[!htbp]
                     \centering
                     \includegraphics[width=0.7\linewidth]{figures/e2program1.png}
                     \caption{实验二(2)-(3)路由1程序运行结果}
                     \label{fig:e2program1}
                 \end{figure}

        \item 实验二(2)-(3)路由2程序运行结果如图~\ref{fig:e2program2}所示。
                \begin{figure}[!htbp]
                    \centering
                    \includegraphics[width=0.7\linewidth]{figures/e2program2.png}
                    \caption{实验二(2)-(3)路由2程序运行结果}
                    \label{fig:e2program2}
                \end{figure}

        \item 实验二(2)-(3)路由3程序运行结果如图~\ref{fig:e2program3}所示。
                \begin{figure}[!htbp]
                    \centering
                    \includegraphics[width=0.7\linewidth]{figures/e2program3.png}
                    \caption{实验二(2)-(3)路由3程序运行结果}
                    \label{fig:e2program3}
                \end{figure}

      \end{enumerate}

   \newpage
   \section{其他}
   \subsection{不足}
      本实验完成的不足之处有:arp缓存表的替换算法应该是随机替换,而代码中实现的是先入先出;arp缓存没有释放缓存包和缓存条目的内存,有内存泄露问题。

   \subsection{疑问}
      为什么实验一和实验二只能ping通入端口,不能ping通出端口?实验二两个主机节点也相互ping不通(router-reference也是如此)。

   \subsection{收获}
      通过对icmp协议和arp协议底层的代码书写,深刻学习了icmp协议和arp协议的机制。

   \subsection{想法}
      mininet软件对于底层测试不够友好,没有集成的GUI界面,测试的自动化程度不够,修改一处代码以后,需要手动在mininet里重启应用,需要手动发包,需要手动使用wireshark查看包,自动化程度太低,制约了生产力。

   \subsection{说明}
     报告文档通过tex程序输出,源文件在代码目录的report目录下;代码可以在Ubuntu下通过make router命令编译链接生成可执行程序;代码仓库地址:\url{https://github.com/mrbian/091M4002HBP}。联系邮箱为:\url{mrbianliunian@outlook.com}。

\end{document}
